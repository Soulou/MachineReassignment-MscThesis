\chapter{The experiments}
\label{experiments}
\lhead{Chapter 3. \emph{Experiments}}

\section{Study of the ability to isolate containers CPU usage using Linux control groups}
\subsection{Goal of the experiment}

Linux containers are sharing the same operating system, they are not fully
isolated as we can see with complete virtual machines. To achieve this
isolation, the control groups (cgroup) of the linux kernel are used to
apply limits on the resource access right of each container.

This experiment aims at studying how these cgroups are working and how
do they actually share the CPU resources among the different containers.

\subsection{Metrics}

\subsubsection{Inputs}

The number of CPUs that an application consumes has to be clearly defined. In
each container, an application developped to consume a given number of CPU
cores will be launched. The source code of the application can be found at
\url{https://github.com/Soulou/msc-thesis-cpu-burn}.

\vspace{1em}

\lstset{language=bash}
\begin{lstlisting}
# Parameter n: Number of core to consume
./msc-thesis-cpu-burn -nb-cpus=<n>
\end{lstlisting}

The second input corresponds to the number of shares a container can access
on the CPUs of the running computer. This number is arbitrary as the shares
are relative to each other. 

\begin{quote}
If a container does not have any cpu share number specified, the default value
is: $1024$
\end{quote}

It is expected that if there are two containers, one with $1024$ cpu shares and
the other with $2048$ CPU shares, the second container will have access to
$2048/1024 = 200\%$ of the resources, for a single CPU: $33\%$ and $66\%$.

\subsection{Setup}

To test the capacity of the isolation by cpu shares, two different environments
will be used. As the result are expected to be relative to the hardware their should
not be any major differences between both, but as a sanity test, it is important
execute it on two différents contexts

The first one my personal laptop, here are its caracteristics:

\begin{itemize}
	\item{CPU: Intel\textregistered \hspace{1pt} Core\texttrademark
	\hspace{1pt} i7-3537U CPU @ 2.00GHz (2 cores with hyperthreading)}
	\item{Memory: 8 GB RAM DDR3}
	\item{Disk: 256GB Solid State Drive}
\end{itemize}

Then we'll study the results of the same experiment on a 4 cores virtual machine
based on an OpenStack cluster:

\begin{itemize}
	\item{CPU: 4 KVM vCPUs}
	\item{Memory: 8 GB RAM}
	\item{Disk: Virtual HDD 80GB}
\end{itemize}

\subsection{Results}

\section{Load balancing of containers using different bin packing algorithms}
\subsection{Goal of the experiment}

\subsection{Source of the workloads}
\subsubsection{Statistical distribution}
\subsubsection{Sample from Google workload}
\subsection{Metrics}
\subsection{Setup}
\subsection{Results}

\subsection{}
