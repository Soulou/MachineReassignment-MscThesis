Operating system-level virtualization, commonly featured as containers, has taken an
important place in the current IT ecosystem. They allow people to deploy
applications with an additional abstraction level. Whatever is the underlying
hardware, or the version of the operating system, a container will always run
the same way.

This technology has increased the flexibility of an infrastructure, and even
more of a cloud based infrastructure. This thesis defines a platform to test
resource allocation algorithms in a concrete environment, not a simulation. The
scope of this work has been reduced to bin packing algorithms which are the
most common for resource allocation and load balancing. Furthermore, the
studied tasks are only web applications, for their ability to be stateless, and
to migrate easily.

The defined platform is based on \textit{Docker} and on the decentralized
cluster management and service discovery tool \textit{Consul}. Different
utilities have been developed. They create the ability of spawning containers
on all the servers of the cluster. Moreover, those containers are gathered
by service. If two containers are part of the service ``abstract``, all the
requests on \texttt{http://abstract.thesis.dev} are shared between them.

Online bin packing and offline bin packing algorithms have been used to
respectively study the resource allocation and the load balancing problems
Experiments have been designed to measure their impact on the performance
of the running processes part of the infrastructure.

\vspace{2em}
\textbf{Keywords}: operating system-level virtualization, containers, load
balancing, resource allocation, bin packing, cloud environment, virtual
machines, web applications
