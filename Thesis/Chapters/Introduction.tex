\chapter*{Introduction}
\addcontentsline{toc}{chapter}{Introduction}
\lhead{\emph{Introduction}}

Optimization of allocation is a common problem to every developer or system
administrator who has to scale software programs in a distributed setting such
as a cloud environment. What is the best distribution for the set of
applications we need to deploy? Everybody is looking for the best
price-performance ratio. This thesis deals with the problem of resource
allocation and gives the design of an experimental infrastructure which can be
considered as an ``enterprise'' infrastructure.

The final goal is to have a convenient platform to perform experiments in order
to evaluate the efficiency of resource allocations algorithms using different
heuristics. The aim of this work is to be able to have an overview of what is
possible and to give recommendations concerning good practices and interesting
algorithms to use in a real world situation.

In the last decade, the hardware has kept improving year
after year, especially processors, which have been improved at doing operations
simultaneously. Infrastructure owners have been struggling to create ways to
split those resources securely and efficiently, so they may be used by several
isolated customers. To answer this need, technologies related to the concept of
virtualization~\cite{virtualisation} have been created, enabling to split
servers into isolated sub-servers, sharing resources from a common host.

\textbf{Amazon Web Service} started commercialising their service of
on-demand virtual servers named \textbf{Elastic Compute Cloud} in 2006. It was
the first actor of a large market, which has since grown hugely. Based on the
concept of ``pay as you go'', their customers are able to adapt, in real
time, their infrastructure consumption, impacting directly the money they are
spending. This attractive feature gave birth to the first cloud services like
\textbf{Dropbox} which was first based on Amazon infrastructure.

An interesting definition of the Cloud Computing has been written by the
National Institute of Standards and Technology~\citep*{nistcloudcomputing}:
\begin{quote}
	“Cloud computing is a model for enabling ubiquitous, convenient, on-demand
	network access to a shared pool of configurable computing resources (e.g.,
	networks, servers, storage, applications, and services) that can be rapidly
	provisioned and released with minimal management effort or service provider
	interaction.”
\end{quote}

This thesis doesn't focus on the allocation of virtual machines in a set of
physical servers, but at a higher level, the allocation of application
containers. Actually, VMs have been used extensively and studied for more than
a decade now, but this is not the only way to achieve virtualization.
Containers are considered as lightweight virtual machines. Instead of being
global (virtualization of hardware and operating system), they focus on the
isolation at the application level. This method is getting more and more
attraction: companies have started requesting infrastructure based on
it~\cite{dockerOverVMWare}.

When using containers, it is a common practice to split its product into
a set of loosely coupled softwares, which are communicating together using
a common communication protocol. The most often, the web (HTTP) is used, and those
services are sending and receiving requests through HTTP APIs\@. One of the main
advantages of those applications is that they are stateless. As a result, it is
much more easy to move them from one server to another, this is why this thesis
will focus on these particular softwares.

The main question which will be answered is: how those services can be managed
when they are containerized and how is it possible to apply resource allocation
methods in a concrete infrastructure to manage all the containers and their
resource consumption?

$\Rightarrow$ In the first chapter, a literature review provides a large
background concerning the motivation behind resource allocation and load
balancing, showing that a lot of different perspectives have been used by
researchers these last years. A particular focus is given to bin packing
algorithms which will be used later in this work. \vspace{5pt} \newline
$\Rightarrow$ Then, the second part provides details about the scope of
the thesis. Containerized web applications features and limits are covered.
\vspace{5pt} \newline $\Rightarrow$ The third part is a preliminary experiment
to measure the ability to share the CPU cores of a server between containers.
\vspace{5pt} \newline $\Rightarrow$ The design of the experimental platform in
a cloud environment is defined in the chapter 4. The different roles of the
servers are explained, as well as the deployment of the complete software suite
in another infrastructure.\vspace{5pt} \newline $\Rightarrow$ Finally, the
last chapter shows different experiments done in the previously created
infrastructure and some discussions concerning the use of bin packing
algorithms.

 
