\chapter{Study of algorithms for Containers allocation and load balancing}
\label{containerloadbalance}

\section{Experimental Setup}

This part will be devoted to the study of the containerized web application
allocation and load balancing in a cloud environment. Before speaking of the
experiments themselves, it is important to define how those experiments have
been setup. The target is the ability to test algorithms using a realistic
infrastructure and not to create a simulation of it.

\subsection{Hardware Infrastructure}

The experiments have been done on a private cloud infrastructure, powered by
Openstack (version: Grizzly). The amount and the capacities of the virtual
machines have been different according to the experiment, but all the instances
are always in the same private network.

This document won't cover how to install Openstack but there are plenty of
tutorials on the web. Moreover this setup can be done in a public cloud
infrastructure like Amazon Web Services EC2, there is no difference. We are
going to use two distincts kind of nodes. The agents execute the
different web applications, and the controller which is the interface to
control the complete infrastructure.

\subsection{Software Infrastructure}
\subsubsection{Operating System}

All the virtual machines are running \textbf{Ubuntu Server 14.04 LTS}, this choice
has been lead by the fact that this Linux distribution is probably the most standard
worldwide and because \textit{python 3.4} is required to run some libraries of the project
to execute the experiments.

The cloud version of the distribution has been chosen\footnote{Download page of
Ubuntu 14.04 Cloud :\url{http://cloud-images.ubuntu.com/releases/14.04/}} if
order to be compatible with Openstack and correctly boot. To add the image to
an openstack cluster, only one command is required:

\vspace{1em}
\begin{lstlisting}
glance add name="ubuntu-trusty" is_public=true \ 
  container_format=ovf disk_format=qcow2 < /path/to/file.img
\end{lstlisting}

To deploy, run, stop and migrate our applications, \textbf{Docker} will be
used. More precisely its REST API. Actually all the requests to Docker are done
through HTTP requests on a unix socket. (\textbf{Docker} is using a unix socket
owned by root for security reasons, to avoid remote access to the host)

\subsubsection{Service discovery}

One of the common difficulties in cloud infrastrucure gathering numerous
virtual machines is the service discovery. It is possible of course to use a
configuration manager to generate static configuration on each node which will
be used by the different services. However this system is static, doesn't scale
well and is not fault resiliant. That is what it is a bad idea to write
anything statically when deploying such infrastructure.

The project \textbf{Consul} has been used to achieve the feature. Consul is
decentralized solution for service discovery based on two protocols. On the one
hand, it is using a gossip protocol to manage the communication between nodes.
This feature let \textbf{Consul} creates a decentralized cluster of servers.
When a new node is available, it just needs to communicate with one node,
whichever it is, to join the complete cluster and get access to the shared
resources. On the second hand, the process is using a consensus algorithm to
elect a leader node on the cluster, which has the responsability to keep the
data consistent. Write operations have to be validated by the leader node, then
spread to the rest of the servers. If the leader node crashes, another node is
automatically elected by the others nodes.

\textbf{Consul} main usage is service discovery, so each node is regitering its
running services to consul which will spread the information among the whole
cluster. That is how services get to know each other.

The application is developped using the \textbf{Go} programming language, so
the installation is trivial. To achieve the installation, whatever is the
operating system, downloading the binary from the website
\url{http://www.consul.io/downloads.html} and executing it enough.  The
configuration of each node service is done through a set of JSON files which
have to be defined in \textbf{Consul} configuration directory.

\subsubsection{Balancer agent}

On each server which has to exectute the web applications, the installation
of the balancer agent is required. It is a HTTP server written using python3.
The source code can be found on GitHub
\footnote{\url{https://github.com/Soulou/msc-thesis-container-balancer-agent}}. The
installation is straightforward.

\vspace{1em}
\begin{lstlisting}
git clone \
  https://github.com/Soulou/msc-thesis-container-balancer-agent
cd msc-thesis-container-balancer-agent
virtualenv -p /usr/bin/python3 .
source bin/activate
pip install -r requirements.txt
\end{lstlisting}

As the agent has to communicate with \textbf{Docker} is should be run as
root:

\vspace{1em}
\begin{lstlisting}
sudo -E python agent.py
\end{lstlisting}

The server has two distinct roles. The first one is to execute instructions
coming from a controller, the interface is an HTTP API. You can find its
documentation in the Annexe A: ~\nameref{app:agent-api}

The other role of the agent is to achieve real time monitoring of the
server itself and of each container running on it. Different threads
starts in parallel with the HTTP Server. The reason why it is necessary
to use separate threads is that at a given time it's not possible to
get some relative data. This is how the different metrics are gathered
by the agent.

For the entire server:

\begin{itemize}
\item{CPU: The interface from the Linux Kernel to read the CPU usage is
\texttt{/proc/stat}. When this virtual file is read, the kernel fills it with
the current information about the CPU usage, the interruptions and the
processes. However those data are cumulative. So each second the data are
fetched, and compared to the CPU usage of the previous second. The data are un
\textit{User Hz}, this unit represent a tick in the user space, 100 of them are
generated per second, so the value shown in this file are close to hundreths of
second.} \item{Memory: This value is easier to access, the kernel provides
\texttt{/proc/meminfo} which contains the real time data usage. There is no
extra work to do in order to the clean pieces of information.} \item{Network
I/O: \texttt{/proc/net/dev} contains all the information related to all the
network interfaces of the server, in this file is displayed the amount of bytes
and packets sent and received by each of them. The values are also cumulative,
that's why the agent has to keep track of them. As a result when a request is
done to get the system usage, the right data can be sent directly and it's not
required to wait 1 second.}
\end{itemize}

For the containers:

\begin{itemize}
\item{CPU: The CPU usage of each container is accounted separately thanks to
the \textit{cpuacct} cgroup feature of the Linux Kernel. The communication
from the userspace is done through a virtual file system located at
\texttt{/sys/fs/cgroup}. As a result, for docker we can find the correct data
at that path: \newline\texttt{/sys/fs/cgroup/cpuacct/docker/:container\_id/cpu.usage}.
As precedently, the value is in \textit{User Hz}, so the process to calculate the
actual CPU usage is similar as the \texttt{/proc/stat} analysis.
\item{Memory: The cgroup \textit{memory} manages the memory usage and limits per
container, it is enough to read
\textit{/sys/fs/cgroup/memory/docker/:container\_id/memory.usage\_in\_bytes} to get
the interesting piece of information.}}
\item{Network I/O: It is a bit more difficult to monitor the network usage of a container,
as the resource management mecanisms is not part of the cgroup, it is another feature
of the Linux Kernel called "network namespace". In order to get access to it, different
steps are required.
\begin{enumerate}
\item{Find the PID of a process in the monitored container by looking in \texttt{/sys/fs/cgroup/:cgroup/docker/:container\_id/tasks}}
\item{Access the network namespace file located in \texttt{/proc/:pid/net/ns}}
\item{Create a link of this namespace to \texttt{/var/run/netns/:container\_id}}
\item{Use IP command to get stats fron the desired namespace \texttt{ip netns exec :container\_id netstat -i}}
\end{enumerate}}
\end{itemize}

To sum up, three distinct threads are running in the Balancer agent, the HTTP server,
the host system monitoring, and the containers monitoring. Those threads are used to
be able to get accurate data instantly, otherwise 1 second should be waited to get
interesting data.

\subsubsection{Balancer controller}

The second main brick of this infrastructure is the controller. The role of this software
is to control all the different agents, to give them the instructions about which container
to start and which container to stop.

The running applications on the cluster are gathered by \textit{service}. Each
of them can contain a variable amount of containers hosted on the different agent.

Moreover, it updates dynamically the routing tables of the different services
running on the infrastructure. If a service called "service-test" owns two containers,
the incoming requests will be routed to both of thoses containers by following a
round-robin algorithm. (C1 - C2 - C1 - C2 - ...).

\textbf{Hipache}\footnote{\url{https://github.com/dotcloud/hipache}} is used as
a front HTTP working as reverse proxy, it is in charge of routing incoming requests
to the different containers.
Why \textbf{Hipache} has been used instead of a more classical server like
\textbf{Apache}. The main reason is that \textbf{Hipache} is dynamically
configurable thanks to different backends. The most common is the
\textbf{redis} backend. \textbf{Redis} is a key-value store with really high
performance as all the dataset stay in memory, and is asynchronously written to
the disk. So when the controller sends request to start or stop a container, it
also connects to a \textbf{redis} instance to update hipache configuration.

The controller also exposes a HTTP API

This component of the infrastructure is critical for the experiment detailed
later in this work.

\subsubsection{Balancer client}
\subsubsection{Deployment}

\section{Deployed applications}

\section{User load simulation}

\section{Offline bin packing algorithms}


\section{Online bin packing algorithms}

\section{Results}
