\clearpage
\section*{Conclusion}
\lhead{\emph{Conclusion}}

The purpose of this work has been to show how containers are a new efficient
way to isolate and deploy web applications. Then, we have seen that is it
possible apply limitation and quotas to limit how a container behaves. The next
step has been to design a software platform, aiming at deploying containers
easily into a cloud environment, but also to be able to move them between all
the available nodes. Furthermore this set of server appplications are gathering
real time metrics about the running containers and servers. With these data,
bin packing algorithms have been made possible and the last chapter of this
thesis has shown different experiments concerning the different way to use them
to allocate resource, consolidate the infrastructure or balance the load of the
servers.
\vspace{1em}

The developed platform has shown that it is able to manage efficiently
containerized web applications. This feature has lead to interesting results
concerning resource allocation. When a new container has to be run, we have
seen that a mix of simple bin packing heuristics is giving some really
interesting results. Combined together best fit and worst got the best ratio
``number of used nodes by number of answered requests per second''
\vspace{1em}

The field of resource allocation is getting more and more important now,
because infrastructure are getting more and more heavily distributed and I
have the feeling I have only scratched the surface of what can be done.
However, I think the developed platform is good enough and could extend
in order to be used for further experiments.
\vspace{1em}

This writing has principaly focused the obtained performance for containers
which were able to run without any limit. This is something which is really
common. A lot of service providers are based on a ``fair use'' policy. In this
case they do their best to run everything, but they can warn or give explicit
quota to user abusing this statement. If we get outside this scope, other
experiment should be done to get new results.
\vspace{1em}

Working in a real environment instead of a simulation changes a lot of settings,
I think that with the developed tools, a lot more testing and experiment
could be done, especially concerning load balancing. Another possible extension
could be some statistical prediction about the future consumption of a service,
based on statistical and probabilistical models. 
\vspace{1em}

A particular effort has been made to make this work reproducible. All the
tools are open-source softwares and protocols standard.  Thanks to
\textbf{Ansible} the environment should be easy to deploy on any
infrastructure, physical, virtual or hybrid.  Even if the obtained results in
another deployment of the platform would be slightly different, it should
follow the same trends as those obtained in this thesis.
