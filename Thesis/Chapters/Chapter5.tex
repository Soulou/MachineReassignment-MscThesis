\chapter{Study of algorithms for Containers allocation and load balancing}
\label{chapt:containerloadbalance}

\section{Experimental applications}

The experimental process has to measure the performance of a set of containers
before and/or after using any resource allocation algorithm. First, we have to
define which applications will be containerized for those operations. Those
applications have to be deployed on all the agent servers and have to behave as
standard web services. Whatever is the language or the framework used (except
PHP), all the dependencies of the applications are loaded on application
startup, so afterwards, there is no more file to read from the disk.  As a
result the disk input/output are really low for most of the applications and it
this metric will not be measured in the following experiments.

The first service which has been used during the experiments is an application which
is calculating $N$ elements of the Fibonacci suite\footnote{\textbf{Docker}
image: \texttt{soulou/msc-thesis-fibo-http-service}}. This web application has
only one endpoint: \texttt{GET /:n}, which returns the $N^{th}$ items, of the
Fibonacci sequence. This application only consumes CPU, the memory footprint is
really negligible.

The second application which is used, has been designed to consume a precise
amount of resource during a specified duration\footnote{\textbf{Docker} image:
\texttt{soulou/msc-thesis/memory-http-service}}. Its endpoint is \texttt{GET
/:memory/:time} with the memory in megabytes and the time in milliseconds. For
instance, \texttt{GET /10/100} will consume precisely 10MB for a duration of
0.1s. Thanks to this application, we can measure precisely how much should
consume the application with a load of $N$ requests per second of a certain
amount of memory during a precise timelapse.
The memory is allocated and initialized. As a result, even if we are requesting
amount of memory, it also generates an important CPU load in the loops for memory
application.

The source code of bother services can be found on GitHub:
\begin{itemize}
\item{\url{https://github.com/Soulou/msc-thesis-fibo-http-service}}
\item{\url{https://github.com/Soulou/msc-thesis-memory-http-service}}
\end{itemize}

\section{Load generation}

An important question when measuring the performance against an infrastructure,
in this case: web services, is the way to generation the requests. Do they
have to reflect a realistic user load, is it possible to do it, is it pertinent?
In this case, we are going to measure raw performance over the cluster, to gather
data about the efficiency of a specific algorithm. When user load simulation algorithm
is used, the benchmark is irregular, and finally getting information about
the real impact of a given algorithm may be more difficult to do.

\subsection{Tools}

To generate web traffic, HTTP requests generators are required. Historically,
the most used tool is part of the \textbf{Apache} web server tools, and is
called Apache Bench \textit{Command line name: ab}, released under the open-source
\textit{Apache Licence}. This utility is not really
used anymore because it is single-threaded which is a very limiting parameter.
Only one CPU core is used, it may not be enough to saturate a target and get a
correct measure of the performance, if the measuring tool is limiting.

The tool which has been mostly used in the scope of this work is named
\textbf{wrk}\footnote{\url{https://github.com/wg/wrk}}. This is a benchmark
tool able to send requests using a given number of connections, used in
different parallel threads. (The opposite of \texttt{ab}, which sends all
the request concurrently using one thread)

Usage example:

\vspace{1em}
\begin{lstlisting}
wrk -c 10 -t 2 -d 1m http://service1.thesis.dev
\end{lstlisting}

This example will send 10 requests concurrently during 1 minute using two threads.
So we are sure that the targeted URL will receive a maximum of 10 request at a
given time.

One reproach which can be made to this tool is that it doesn't adjust the
number of threads automatically. By default, two threads are used, but if we
need more parallel connections, the user has to define it by hand. But is most
of the case, giving a thread per core on the underlying computer is the
best practice.

\subsection{Different kinds of load}

Combining \textbf{wrk} and the two previously defined web application, different
kinds of load can be generated. Thanks to the "memory" HTTP service, it is even
possible to emulate different kinds of application.

In some cases, web applications are micro-services and their job is to do a small
particular task. Commonly, requests are really quick and the treatment of each of
them has a low memory footprint. However as those HTTP requests may be numerous,
the overall processor and memory usage can be important.

In out infrastructure, one container of such a service may be represented by instances
of the memory services with requests using a few megabytes of memory, are done
really quickly: \texttt{http://micro-service.thesis.dev/5/100}. All the requests to this
endpoint are compliant with the previous constraints. The following data show how 
such application behave in an environment where they don't have to share resources\footnote{All the measure
have been done 5 times, and the given result is  the average}:

\begin{itemize}
	\item{One instance: \newline
	\texttt{wrk -c 20 'http://memory-service.thesis.dev/5/100'} - 82.05req/s - CPU 106\%, 24MB \newline
	\texttt{wrk -c 40 'http://memory-service.thesis.dev/5/100'} - 93.74req/s - CPU 112\%, 22MB \newline
	There is no big difference when running 20 or 40 connections in parallel, so we can assume that we have
	reached the maximum capacity of the instance. As announced previously, the memory usage is really low, but
	the application is using one complete core.\vspace{1em}}
	\item{Two instances on two different hosts: \newline
	\texttt{wrk -c 20 'http://memory-service.thesis.dev/5/100'} - 123.23req/s - CPU 109\% | 80\%, 15MB | 14MB \newline
	\texttt{wrk -c 40 'http://memory-service.thesis.dev/5/100'} - 165.77req/s - CPU 117\% | 97\%, 17MB | 31MB \newline
	The exepectations were that two instances would be able to execute twice the number of requests compared to
	the previous case. With 20 connections, it is not the case, but with 40, the service has been able to execute
	165 requests per second, which is more or less the double as previously. 20 connections was not enough. This
	lack can be seen by the fact that the CPU usage of one of the instances is not one complete core but 80\%.}
	\item{Two instances on the same host: \newline
	\texttt{wrk -c 20 'http://memory-service.thesis.dev/5/100'} - 101.11req/s - CPU 97\% | 98\%, 17MB | 18MB \newline
	\texttt{wrk -c 40 'http://memory-service.thesis.dev/5/100'} - 127.77req/s - CPU 115\% | 117\%, 18MB | 17MB \newline
	The instances have 2 vCPUs, so theoretically two instances of a same service on one node should be able to run
	at maximumal performance, but in practice, the results are not as good, a ratio of 1.5 has been reached using
	40 parallel connections. This is a perfect illustration of a "bad neighbour" situation, the performance of
	each container is jeopardized by the other instance. That is something which can be (at least partially) solved
	with scheduling algorithms, to avoid as much as possible this case.}
\end{itemize}

Note: The values of CPU usage are often above 100\% when an application is
using on core.  The amount of \textit{User HZ} is read every second, so it may
be an inacurracy of this timer, or it may be that the time used by the kernel
to schedule processes is taken into account. As a result, the process uses
100\% and the kernel uses some CPU time on another core.

\section{Online bin packing algorithms}

An online bin packing algorithm as defined in the first chapter of this
work:~\nameref{litreview}, is an algorithm which has to pack new items into
bins without having the knowledge of what is already present in the bins. So
bins have a remaining capacity, and each new items is known or partially known.
In some cases, the new items are completely unknown.

In this infrastructure there are two different cases which can be distinguished:

\begin{itemize}
\item{The new container is part of a new service: there is no information about
the number of incoming requests, and what is the memory footprint of the new process}
\item{The new container is from an existing service: in this case, the CPU consumption
and memory usage can be estimated by looking the other containers of a similar service.
For instance, if two containers of \texttt{service1} are running using both 20\% of CPU,
it is probably true that the consumption of the new container will be capped at 20\%}
\end{itemize}

In the first case, it

\begin{verbatim}

Round-Robin

Service1 Requests/sec:     16.71 16.69 16.91 16.74 16.67 16.48
Service2 Requests/sec:     22.92 23.76 23.16 23.87 23.31 23.20
Service3 Requests/sec:     27.68 28.30 28.13 28.69 27.18 28.83

Random

Service1 Requests/sec:     15.21 13.17 17.05 15.05 15.33
Service2 Requests/sec:     21.78 12.83 23.14 19.73 19.85
Service3 Requests/sec:     30.33 19.79 28.68 28.72 24.85

\end{verbatim}


\section{Offline bin packing algorithms}
\section{Results}


