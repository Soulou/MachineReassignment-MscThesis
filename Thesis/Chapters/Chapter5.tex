\chapter{Study of algorithms for Containers allocation and load balancing}
\label{chapt:containerloadbalance}

\section{Experimental applications}

To achieve experiments, applications have to be deployed. Those applications
have to behave as standard web services. Whatever is the language or the
framework used (except PHP), all the dependencies of the applications are
loaded on application startup, so afterwards, there is no more file to read
from the disk.  As a result the disk input/output are really low for most of
the applications.

Two main applications will be used during the experiment, an application which
is calculating $N$ elements of the Fibonacci suite\footnote{\textbf{Docker}
image: \texttt{soulou/msc-thesis-fibo-http-service}}. This weh application has
only one endpoint: \texttt{GET /:n}, which returns the $N^{th}$ items, of the
Fibonacci sequence. This application only consumes CPU, the memory print is
really small.

The second application which is used, has been designed to consume a precise
amount of resource during a specified duration\footnote{\textbf{Docker} image:
\texttt{soulou/msc-thesis/memory-http-service}}. Its endpoint is \texttt{GET
/:memory/:time} with the memory in megabytes and the time in milliseconds. For
instance, \texttt{GET /10/100} will consume precisely 10MB for a duration of
0.1s. Thanks to this application, we can measure precisely how much should
consume the application with a load of $N$ requests per second of a certain
amount of memory during a precise timelapse.

\section{Load generation}

To generate load against web applications, web requests generators are
required.  Historically, the most used tool is part of the \textbf{Apache} web
server tools, and is called Apache Benchmark \textit{Command line name: ab}. In
the following experiment, another utility is going to be used:
\textbf{wrk}\footnote{\url{https://github.com/wg/wrk}}. This is a benchmark
tool able to send requests using a given number of connections, used in different
parallel threads.



\section{Online bin packing algorithms}

\begin{itemize}
\item{Service1: }
\end{itemize}

\begin{verbatim}

Round-Robin

Service1 Requests/sec:     16.71 16.69 16.91 16.74 16.67 16.48
Service2 Requests/sec:     22.92 23.76 23.16 23.87 23.31 23.20
Service3 Requests/sec:     27.68 28.30 28.13 28.69 27.18 28.83

Random

Service1 Requests/sec:     15.21 13.17 17.05 15.05 15.33
Service2 Requests/sec:     21.78 12.83 23.14 19.73 19.85
Service3 Requests/sec:     30.33 19.79 28.68 28.72 24.85

\end{verbatim}


\section{Offline bin packing algorithms}
\section{Results}


