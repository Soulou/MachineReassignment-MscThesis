
\chapter{Client command line tool documentation}
\lstset{language=Bash, basicstyle=\footnotesize\ttfamily}
\label{app:client-doc}

\vspace{1em}

\begin{itemize}
\item{Start a container
\begin{lstlisting}
./client.py start [--image <image name>] <service name>

##
# The Image parameter correspond to the docker image which should be
# run on the agent, as at that time, the identify of this agent is not
# known, take care that all the agents have the requested docker image.
# By default a fibonacci web service is executed.
#
# The service name is the name of the group of containers that the
# newly created process will be appended, if the group doesn't exist,
# it is created.
#
# Example
#

./client.py start --image soulou/memory-http-service service1

-> Host: 192.168.114.13
-> ID: e6bbb8853678b67882ea6919c559d80a20
-> Service: service1
-> Image: soulou/msc-thesis-memory-http-service
-> Port: 49342
\end{lstlisting}}
\item{Stop a container
\begin{lstlisting}
./client.py stop --service <service_name> | --all <host> | <host> <ID>

##
# Three different options are available to stop a container.
# * All the container of a service ban be stopped
# * All the container of a precise server may be stopped
# * A precise container can be shutdown by giving its host and its ID
#
# Example
#

./client.py stop --service <service_name>

Container f6fe748fbf from 192.168.114.14 has been stopped
Container e7dc0f68f3 from 192.168.114.17 has been stopped
Container 6a28e4b2ab from 192.168.114.15 has been stopped
\end{lstlisting}}
\item{Get status of the infrastructure
\begin{lstlisting}
./client.py status

##
# Give the information about all the running containers
#
# Output example
#

+----------------+-------+-----------+
| Node           |  Port |  Service  |
+----------------+-------+-----------+
| 192.168.114.14 | 49196 |  service3 |
| 192.168.114.14 | 49195 |  service3 |
| 192.168.114.13 | 49342 |  service1 |
+----------------+-------+-----------+

+---------------------------------------+
|                 Image                 |
+---------------------------------------+
| soulou/msc-thesis-memory-http-service |
| soulou/msc-thesis-memory-http-service |
| soulou/msc-thesis-memory-http-service |
+---------------------------------------+

+-----------------+---------------------+
|        ID       |      Created At     |
+-----------------+---------------------+
| 3c9345c19e00d21 | 2014-07-28 10:51:34 |
| 9a2e2b22448f766 | 2014-07-28 10:51:33 |
| e6bbb8853678b67 | 2014-07-29 15:52:15 |
+-----------------+---------------------+
\end{lstlisting}}
\item{Migrate an application to another node
\begin{lstlisting}
./client.py migrate <host> <id>

##
# This operation asks the controller to migrate a container
# to another node
#
# Example
#

./client.py migrate 192.168.1.3

Container 192.168.1.3 - 9a2e2b2244 -> 192.168.1.18 - 406d5ce46d
\end{lstlisting}}
\item{Get node or nodes status
\begin{lstlisting}
./client.py node_status <host>
./client.py nodes_status

##
# Get the system metrics for on particular node or all the nodes
#

./client.py nodes_status
[ '192.168.114.19': {'cpus': {'cpu0': 0, 'cpu1': 0},
                    'free_memory': 3736440,
                    'memory': 4048292,
                    'nb_containers': 0,
                    'net': {'docker0': {'rx': 0, 'tx': 0},
                            'eth0': {'rx': 3728, 'tx': 1858},
                            'eth1': {'rx': 0, 'tx': 0},
                            'lo': {'rx': 0, 'tx': 0},
                            'veth2074': {'rx': 0, 'tx': 0},
                            'vetha4f2': {'rx': 913, 'tx': 1796},
                            'vethc390': {'rx': 0, 'tx': 0},
                            'vethd466': {'rx': 633, 'tx': 964},
                            'vethec34': {'rx': 0, 'tx': 0}}}}
]
\end{lstlisting}}
\end{itemize}
